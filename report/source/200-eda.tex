\subsection{Brief overview}

To start off, we summed up the total sales of all shops and items combined to get a feel of potential seasonal outliers as well as a general trend.

\begin{figure}[h]
  \centering
  \includegraphics[width=0.9\linewidth]{external_content/graphs/total_sales.png}
  \captionsetup{justification=centering}
  \captionof{figure}{Total sales of the company}
  \label{fig:total_sales}
\end{figure}


We can observe that the sales peak in the month of December. This could be due to the increasing demand of gifts and disposable income from the population, but it could also indicate special Christmas sales which are common at this time of the year. Additionally, we observe a slight decline in demand over the timespan of the dataset.

To compare and validate the above statements, we plotted the total revenue of the company:

\begin{figure}[h]
  \centering
  \includegraphics[width=0.9\linewidth]{external_content/graphs/total_revenue.png}
  \captionsetup{justification=centering}
  \captionof{figure}{Total revenue of the company}
  \label{fig:total_revenue}
\end{figure}

The seasonality of the data is clearly confirmed. Meanwhile, the downwards trend does not appear to be of particular relevance. The revenue stream has a constant trend over the years while having fewer sales. We should hereby be aware of the trend of having fewer sales with more expensive items.
